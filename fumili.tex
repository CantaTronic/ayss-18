
FUMILI is one of the first minimizers included into ROOT release.
It has been showing it's reliability, stability and high convergence rate while it had been being used by scientific community for decades.

The greedy minimization algorithm which is employed in FUMILI was first proposed and developed at JINR by I.\,N.~Silin and V.\,S.~Kurbatov.

FUMILI provides an optimal solution for $\chi^2$-like functionals \eqref{xisq_func} employing linearization:
\begin{equation}
\label{xisq_func}
F(x) = \sum_{k=1}^K f_k^2(x) = \sum_{k=1}^K\left(\frac{Y_k - T_k(x)}{\sigma_k} \right)^2,
\end{equation}
where $Y_k$ are measured values with errors $\sigma_k$, $k \in [1, K]$, and $T_k(x)$ are the values predicted by the model, depending on some parameters $x = \{x_1, \ldots, x_n\}$.

% {Linearization method in minimizing $\chi^2$-like functionals}
The second derivative $\displaystyle\frac{\partial^2 F}{\partial x_i x_j}$ could be found the following way:
\begin{equation}
\begin{aligned}
%  \label{linear}
\frac{\partial^2 F}{\partial x_i \partial x_j} &= \frac{\partial}{\partial x_i}\frac{\partial}{\partial x_j} \sum_{k=1}^K f_k^2(x) = \frac{\partial}{\partial x_i}\sum_{k=1}^K 2 f_k(x) \frac{\partial f_k(x)}{\partial x_j} = \\
&= 2 \sum_{k=1}^K \left( \frac{\partial f_k(x)}{\partial x_i}\frac{\partial f_k(x)}{\partial x_j} + f_k(x) \frac{\partial^2 f_k(x)}{\partial x_i \partial x_j} \right)
\end{aligned}
\end{equation}
\emph{Linearization} means discarding the second term $\displaystyle f_k\frac{\partial^2 f_k}{\partial x_i \partial x_j}$ employing second derivatives, that is considered small in comparison to the first one $\displaystyle\frac{\partial f_k}{\partial x_i}\frac{\partial f_k}{\partial x_j}$.

Its main benefit is that the error matrix for a linearized functional is always positively defined, and thus each step leads to a minimum.


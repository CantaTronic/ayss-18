Распространенной ситуацией в ФЭЧ является выяснение параметров вычислительной модели путем фитирования экспериментальных данных. Как правило, вычисления подобных зависимостей являются достаточно тяжёлыми и не точными, поскольку размерность уравнений, как правило, является очень высокой. Для снижения размерности уравнений мы можем использовать данные о кинематике реакции, выраженные в форме уравнений связи.

Например, могут быть использованы закон сохранения массы или инвариант Лоренца.

В общем случае данные о кинематике реакции могут быть записаны in form of equations \eqref{eq:constr}:
\begin{equation}
\label{eq:constr}
\left\{
\begin{aligned}
\phi_1(P) &= 0,\\
\cdots\\
\phi_n(P) &= 0,
\end{aligned}
\right.
\end{equation}
where $P$ is a vector of parameters.
в большинстве случаев данные уравнения связи являются нелинейными и сложными, и сокращение размерности системы \eqref{eq:constr} путем непосредственного решения оказывается невозможным или не имеющим смысла.

% % constrMinIntro.tex
% The problem of minimizing functionals with constraints~\eqref{eq:constr} arises, for example, in the task of kinematic fitting. This problem traces back to early sixties, when first solutions for the problem were proposed~\cite{b1,b2}.
% Первые решения в области минимизации со связями в задачах кинематического фита были предложены в начала 60-х. Классическим, до сих пор широко применимым методом здесь является метод множителей Лагранжа (ссылка), который используется, например в (ссылка). Специалистами ОИЯИ (ссылка) примерно в те же годы было предложено применение минимизации с использованием штрафных функций.
% Подход, предложенный одним из авторов доклада в 90-е (ссылка) [очень крутой - TODO: прописать, почему.]

% Первые идеи по преодолению подобных сложностей в задачах кинематического фита были предложены в 60-е.
The first ideas on overcoming such complexities in the problems of kinematic fitting were proposed in the sixties.
% Солмицом и Бёрджем \cite{b1} было предложено применение метода множителей Лагранжа (см. далее), который до сих пор находит широкое применение в задачах подобного типа \cite{b4}. %сюда же можно ещё ссылки на эксперименты, там тоже сплошнй Лагранж
Solmitz and Berge in~\cite{b1} proposed to use the method of Lagrange multipliers (see below), which is still widely used in problems of this type~\cite{b4}. % here you can refer to more experiments, they are also using solid Lagrange
% Примерно в то же время специалистами ОИЯИ было предложено применение минимизации с тяжёлым членом \cite{b5} (будет описано далее) и разработан пакет для минимизации $\chi^2$-функционалов FUMILI~\cite{fum_1st}.
Approximately at the same time JINR specialists proposed the use of minimization with the penalty function~\cite{b5}, described below, and developed a package for minimizing the $\chi^2$-like functionals FUMILI~\cite{fum_1st}.
% Позже, одним из соавторов данной работы, было предложено \cite{b6} использование метода устранения производных для кинфита со связями.
Later, one of the co-authors of this work suggested the method of elimination of differentials for kinematic fitting with constraints~\cite{b6}.

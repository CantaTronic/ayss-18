% Method of Lagrange multipliers
These solutions used a method of Lagrange multipliers~\cite{b3}. It still remains the most widely used method for kinematic fitting, see e.g.~\cite{b4}.

\begin{itemize}
\item First proposed at early sixties, see e.\,g.\ J.\,P.~Berge, F.\,T.~Solmitz, H.\,D.~Taft, Rev.\ Sci.\ Instr.\ 32 (1961) 538;
\item Uses \emph{Lagrange multipliers} $\lambda_i$, obtained from the equations
\[\frac{\partial\Psi}{\partial x_1} = \frac{\partial\Psi}{\partial x_2} = \cdots = \frac{\partial\Psi}{\partial x_n} = 0,\]
where $\displaystyle\Psi(x) = F(x) + \sum_{i=1}^m\lambda_i\phi_i(x);$
\item Still the most widely used method for kinematic fitting, see e.\,g.\ KWFIT package \texttt{http://www.phys.ufl.edu/\textasciitilde{}avery/kwfit/}.
\end{itemize}

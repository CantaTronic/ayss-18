% % Method of Lagrange multipliers
% These solutions used a method of Lagrange multipliers~\cite{b3}. It still remains the most widely used method for kinematic fitting, see e.g.~\cite{b4}.
%
% \begin{itemize}
% \item First proposed at early sixties, see e.\,g.\ J.\,P.~Berge, F.\,T.~Solmitz, H.\,D.~Taft, Rev.\ Sci.\ Instr.\ 32 (1961) 538;
% \item Uses \emph{Lagrange multipliers} $\lambda_i$, obtained from the equations
% \[\frac{\partial\Psi}{\partial x_1} = \frac{\partial\Psi}{\partial x_2} = \cdots = \frac{\partial\Psi}{\partial x_n} = 0,\]
% where $\displaystyle\Psi(x) = F(x) + \sum_{i=1}^m\lambda_i\phi_i(x);$
% \item Still the most widely used method for kinematic fitting, see e.\,g.\ KWFIT package \texttt{http://www.phys.ufl.edu/\textasciitilde{}avery/kwfit/}.
% \end{itemize}

% Метод множителей Лагранжа является классическим подходом, нашедшим свое широкое применение в различных областях деятелньости человека.
The method of Lagrange multipliers is a classical approach that has found its wide application in various areas of human activity.
% Задачу \eqref{track_fit}, \eqref{eq:constr} предлагается решать путем нахождения экстремума функции Лагранжа:
It is proposed to solve the task \eqref{track_fit}, \eqref{eq:constr} by finding the extremum of the Lagrange function
\begin{equation}
\label{lagr}
\mathcal{L}(\boldsymbol{p}, \boldsymbol{c}, \boldsymbol{\lambda}) = F(\boldsymbol{p}, \boldsymbol{c}) + \sum_{k=1}^{m} \lambda_k \phi_k(\boldsymbol{p}),
\end{equation}
% необходимым условием которого является равенство нулю ее частных производных по переменным $P_i, ~i \in [0, n_p]$ и неопределенным множителям $\lambda_k$:
that implies the necessary condition of vanishing of its partial derivatives with respect to the variables $p_i$, $i \in [0, n]$ and the undefined multipliers $\lambda_k$:
\[\frac{\partial \mathcal{L}}{\partial p_1} = \frac{\partial \mathcal{L}}{\partial p_2} = \cdots = \frac{\partial \mathcal{L}}{\partial p_{n}} = \frac{\partial \mathcal{L}}{\partial \lambda_1} = \cdots = \frac{\partial \mathcal{L}}{\partial \lambda_{m}} = 0,\]
where $n$ is the number of unknown parameters.
% где $n_p$ - число искомых параметров.
Further, the resulting system can be solved using the Gauss' method or Cramer's rules.
% Далее полученная система может быть решена с помощью метода Гаусса или используя формулы Крамера.

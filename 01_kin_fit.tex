% Распространенной ситуацией в ФЭЧ является выяснение параметров вычислительной модели путем фитирования экспериментальных данных. Как правило, вычисления подобных зависимостей являются достаточно тяжёлыми и не точными, поскольку размерность уравнений, как правило, является очень высокой. Для снижения размерности уравнений мы можем использовать данные о кинематике реакции, выраженные в форме уравнений связи.
%
% Например, могут быть использованы закон сохранения массы или инвариант Лоренца.
%
% В общем случае данные о кинематике реакции могут быть записаны in form of equations \eqref{eq:constr}:
% \begin{equation}
% \label{eq:constr}
% \left\{
% \begin{aligned}
% \phi_1(P) &= 0,\\
% \cdots\\
% \phi_n(P) &= 0,
% \end{aligned}
% \right.
% \end{equation}
% where $P$ is a vector of parameters.
% в большинстве случаев данные уравнения связи являются нелинейными и сложными, и сокращение размерности системы \eqref{eq:constr} путем непосредственного решения оказывается невозможным или не имеющим смысла.

A track reconstruction  problem in general can be understood as a problem of multidimensional linear regression.
Here, for every track $j$ we are looking for a such initial momenta $\boldsymbol{p}^j$ that the reconstructed coordinates of detector 
hits $\hat{c}_i^j(\boldsymbol{p}^j)$, where $i$ is a hit number, would be as close as possible to the registered hit coordinates $c_i^j$, 
taking into account the known detector uncertainties $\delta c_i^j$.
% Задача восстановления трека частицы (фитирования) в общем случае представляет собой задачу многомерной линейной реграссии: для каждого из треков $j$ требуется найти изначальные импульсы $P^j$, такие, чтобы теоретически вычислимое положение точек трека $\hat{c}_i^j(P^j)$, где $i$ - номер координаты точки, максимально приближалось к фактическому распределению точек $c_i^j$, при этом учитывая известные ошибки детектора $\delta c_{i^j}$.
% The actual problem we are solving is the minimization problem for the functional
In the over words, the tasks to be solved here is a minimization of the functional:
% Фактически, решается задача минимизации функционала:
\begin{equation}
\label{track_fit}
F(\boldsymbol{p}^j, c_i^j) = \sum_i \left(\frac{c_i^j - \hat{c}_i^j(\boldsymbol{p}^j)}{\delta c_i^j}\right)^2.
\end{equation}
Further we would omit the indices $i$, $j$ for simplicity, so that $\boldsymbol{c}$ and $\boldsymbol{p}$ would be vectors of all hits and momentum components correspondingly.
% Здесь и далее для простоты будем опускать коэффициенты $i,~j$.

Often one could get an additional information from reaction kinematics in a form of the energy-momentum conservation laws
% Зачастую известна дополнительная информация о кинематике реакции в виде законов сохранения энергии и импульса
\begin{equation}
\label{cons_full}
\sum\mathcal{P}_\mathrm{initial} = \sum\mathcal{P}_\mathrm{final}
\end{equation}
or, in case of one of the final particles is not registered,
\begin{equation}
\label{cons_miss}
\displaystyle\left|\sum\mathcal{P}_\mathrm{initial} - \sum\mathcal{P}_\mathrm{final}\right|^2 = M_X^2,
\end{equation}
where $\mathcal{P}$ is a four-momentum of a particle, and $M_X$ is a reaction missing mass.

Using \eqref{cons_full}, \eqref{cons_miss} allows us to narrow down the multidimensional domain of functional \eqref{track_fit} 
and improves the precision of the result.
% Её использование позволяет сузить многомерную область определения данного функционала и положительно влияет на точность получаемого решения.
A procedure that takes the kinematics constraints into account is called kinematic fitting.
This method is widely employed in high energy physics, e.g.\ in experiments like 
% Выполняемый с учётом данных условий фит называется кинематическим и находит широкое применение в ФЭЧ, например, в таких экспериментах как
BES-III~\cite{BESIII}, % http://iopscience.iop.org/article/10.1088/1674-1137/34/2/009/meta
CLAS~\cite{CLAS}, % https://scholar.google.ru/scholar?cites=6441758058779809394&as_sdt=2005&sciodt=0,5&hl=ru
CMS~\cite{CMS}. % http://cds.cern.ch/record/926540

In this paper we are going to cover a Fumili-based framework for kinematic fitting we have developed, which is currently used for data analysis in the ANKE experiment (Jülich, Germany)~\cite{anke}.
% В данной статье мы собираемся рассказать об опыте разработки Fumili-based фреймвока для кинематического фита, в настоящий момент применяемого авторами для кинематического фита в эксперименте ANKE (Jülich, Germany)~\cite{anke}.

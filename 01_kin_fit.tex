% Распространенной ситуацией в ФЭЧ является выяснение параметров вычислительной модели путем фитирования экспериментальных данных. Как правило, вычисления подобных зависимостей являются достаточно тяжёлыми и не точными, поскольку размерность уравнений, как правило, является очень высокой. Для снижения размерности уравнений мы можем использовать данные о кинематике реакции, выраженные в форме уравнений связи.
%
% Например, могут быть использованы закон сохранения массы или инвариант Лоренца.
%
% В общем случае данные о кинематике реакции могут быть записаны in form of equations \eqref{eq:constr}:
% \begin{equation}
% \label{eq:constr}
% \left\{
% \begin{aligned}
% \phi_1(P) &= 0,\\
% \cdots\\
% \phi_n(P) &= 0,
% \end{aligned}
% \right.
% \end{equation}
% where $P$ is a vector of parameters.
% в большинстве случаев данные уравнения связи являются нелинейными и сложными, и сокращение размерности системы \eqref{eq:constr} путем непосредственного решения оказывается невозможным или не имеющим смысла.

Задача восстановления трека частицы (фитирования) в общем случае представляет собой задачу многомерной линейной реграссии: для каждого из треков $j$ требуется найти изначальные импульсы $P^j$, такие, чтобы теоретически вычислимое положение точек трека $\hat{c}_i^j(P^j)$, где $i$ - номер координаты точки, максимально приближалось к фактическому распределению точек $c_i^j$, при этом учитывая известные ошибки детектора $\delta c_{i^j}$. Фактически, решается задача минимизации функционала:
\begin{equation}
\label{track_fit}
F(P^j, c_i^j) = \sum_i \frac{(c_i^j - \hat{c}_i^j(P^j))^2}{{\delta c_i^j}^2}.
\end{equation}
Здесь и далее для простоты будем опускать коэффициенты $i,~j$.

Зачастую известна дополнительная информация о кинематике реакции в виде законов сохранения энергии и импульса
\begin{equation}
\label{cons_full}
\sum E_\mathrm{initial} = \sum E_\mathrm{final},~\sum\vec{P}_\mathrm{initial} = \sum\vec{P}_\mathrm{final};
\end{equation}
или
\begin{equation}
\label{cons_miss}
\displaystyle\left|\sum\boldsymbol{P}_\mathrm{initial} - \sum\boldsymbol{P}_\mathrm{final}\right|^2 = M_X^2,
\end{equation}
where $E$ is the energy of a particle, $\vec{P}$ and $\boldsymbol{P}$ are its three- and four-momenta correspondingly, and $M_X$ is the reaction missing mass.

Её использование позволяет сузить многомерную область определения данного функционала и положительно влияет на точность получаемого решения. Выполняемый с учётом данных условйи фит называется кинематическим и находит широкое применение в ФЭЧ, например, в таких экспериментах как
BES-III, % http://iopscience.iop.org/article/10.1088/1674-1137/34/2/009/meta
CLAS, % https://scholar.google.ru/scholar?cites=6441758058779809394&as_sdt=2005&sciodt=0,5&hl=ru
CMS. % http://cds.cern.ch/record/926540


В данной статье мы собираемся рассказать об опыте разработки Fumiliu-based фреймвока для кинематического фита, в настоящий момент применяемого авторами для кинематического фита в эксперименте ANKE (Jülich, Germany)~\cite{anke}.

% \textcolor{red}{TODO: привести к общему знаменателю обозначения в формуле с обозначениями в остальной статье!!!}

% обозначим через $np$ число параметров, от которых зависит функционал \eqref{track_fit}, $nc$ - xbckj
% Вблизи фиксированной точки $P_0$ функционал \eqref{track_fit} может быть представлен степенным рядом
In a neighborhood of a fixed point $\boldsymbol{p}^*$ functional \eqref{track_fit} can be represented by a power series
\begin{equation}\label{func_diff}
F(\boldsymbol{p}^*+\Delta\boldsymbol{p}) = F(\boldsymbol{p}^*) + \sum_{i=1}^n \frac{\partial F(\boldsymbol{p}^*)}{\partial p_i} \Delta p_i + \frac{1}{2}\sum_{i=1}^n\sum_{j=1}^n \Delta p_i \frac{\partial^2 F(\boldsymbol{p}^*)}{\partial p_i \partial p_j} \Delta p_j
\end{equation}
and constraints \eqref{eq:constr} by
\begin{equation}\label{constr_diff}
\phi_k(\boldsymbol{p}^*+\Delta\boldsymbol{p}) = \phi_k(\boldsymbol{p}^*) + \sum_{i=1}^n \frac{\partial \phi_k(\boldsymbol{p}^*)}{\partial p_i}\Delta p_i = 0
% = \phi(x_0) + D(x_0) \Delta x.
\end{equation}
Equations (\ref{func_diff}--\ref{constr_diff}) can be rewritten in the matrix form as:
% Данную формулу можно переписать в матричном виде как:
\begin{align}
\label{eq_diff1}
F(\boldsymbol{p}^*+\Delta\boldsymbol{p}) &= F(\boldsymbol{p}^*) + G(\boldsymbol{p}^*) \Delta\boldsymbol{p} + \frac{1}{2}\Delta\boldsymbol{p}^T Z(\boldsymbol{p}^*) \Delta\boldsymbol{p},\\
\label{eq_diff2}
\boldsymbol{\phi}(\boldsymbol{p}^*+\Delta\boldsymbol{p}) &= \boldsymbol{\phi}(\boldsymbol{p}^*) + D(\boldsymbol{p}^*) \Delta\boldsymbol{p} = 0,
\end{align}
where $G(\boldsymbol{p}^*)$ is the gradient, and $Z(\boldsymbol{p}^*)$ is the Hesse matrix.
% где $G(P_0)$ это градиент, а $Z(P_0)$ - матрица Гессе.
Here the rectangular matrix $D(\boldsymbol{p}^*)$ has $m$ rows and $n$ columns, since we have $n$ parameters and $m$ constraints.

The vector $\Delta\boldsymbol{p}$ could be split into $\Delta\boldsymbol{p}_c$ that has $m$ components, and $\Delta\boldsymbol{p}_f$ that has $n-m$ components; the same could be done with the matrix $D$. Then \eqref{eq_diff2} could be rewritten as
\begin{equation}\label{phi_split}
\boldsymbol{\phi}(\boldsymbol{p}^*) + D_c(\boldsymbol{p}^*) \Delta\boldsymbol{p}_c + D_f(\boldsymbol{p}^*) \Delta\boldsymbol{p}_f = 0.
\end{equation}
Since \eqref{phi_split} is a linear equation, it could be solved in relation to $\Delta\boldsymbol{p}_c$, resulting in
\begin{equation}\label{P_c}
\Delta\boldsymbol{p}_c = \boldsymbol{v} + M \Delta\boldsymbol{p}_f.
\end{equation}
Returning to initial functional \eqref{eq_diff1}, we could employ \eqref{P_c} to eliminate the sub-vector $\Delta\boldsymbol{p}_c$, and obtain a similar functional, in contrast depending on only $n-m$ increments $\Delta\boldsymbol{p}_f$:
\begin{equation}\label{func_fin}
F(\boldsymbol{p}^*+\Delta\boldsymbol{p}) = \tilde F(\boldsymbol{p}^*) + \tilde G(\boldsymbol{p}^*) \Delta\boldsymbol{p}_f + \frac{1}{2}\Delta\boldsymbol{p}_f^T \tilde Z(\boldsymbol{p}^*) \Delta\boldsymbol{p}_f.
\end{equation}

% The problem of minimizing functionals with constraints arises, for example, in the task of kinematic fitting.
% % \begin{block}{Kinematic fitting}
% \begin{itemize}
% \item Tracking detectors provide the coordinates of the triggered sensitive elements along with their errors;
% \item Track-finding involves fitting the particle trajectories to these coordinates;
% \item Sometimes, when the reaction channel is known, the additional information on kinematics could be utilized in terms of
% % \begin{description}
% \item[conservation laws:] $\sum E_\mathrm{initial} = \sum E_\mathrm{final}$, $\displaystyle\sum\vec{P}_\mathrm{initial} = \sum\vec{P}_\mathrm{final}$;
% \item[missing mass:] $\displaystyle\left|\sum\boldsymbol{P}^{(4)}_\mathrm{initial} - \sum\boldsymbol{P}^{(4)}_\mathrm{final}\right|^2 = M_X^2$;
% \end{itemize}
%
% % \end{description}
% This is called \emph{kinematic fitting}.

% The minimum search for the functional \eqref{track_fit} could be formulated as a conditional-minimization problem in canonical form, 
% taking into account the known 

The conservation laws (\ref{cons_full}--\ref{cons_miss}), could be expressed as
% Поставленная задача нахождения минимума функционала \eqref{track_fit} может быть сформулирована как задача условной минимизации в канонической форме, с учетом известных законов сохранения \eqref{cons_full}--\eqref{cons_miss}, выраженных в виде:
\begin{equation}
\label{eq:constr}
\phi_k(\boldsymbol{p}) = 0, 1 \leqslant k  \leqslant m,
\end{equation}
where $m$ is a number of constraints.

Then \eqref{track_fit}, \eqref{eq:constr} is a constraint minimization problem written into the augmented form.

The cases of interest is those where the system \eqref{track_fit},~\eqref{eq:constr} is non-linear and could not be practically solved in a straingh forward way, with taking in account practical time and memory limitations. % в условиях заданных внешних ораничений?
% Предмет интереса представляют случаи, когда система \eqref{eq:constr} является нелинейной и её решение ``в лоб'' является невозможным либо нецелесообразным в условиях заданных внешних ораничений.
% So we need not to lose important information and make calculations as accurately as possible, at the same time keeping within the limitations of time and computational intensity of the work.
% Возникает необходимость не потерять важную информацию и провести расчёты максимально точно, при этом уложившись в ограничения по времени и трудоемкости работы.

% Отличительной особенностью задач кинематического фита, как задачи минимизации со связями является то, что функционал (1) здесь зависит не просто от параметров, но от функции этих параметров.
% The distinctive feature of the kinematic fitting problems is that the functional \eqref{track_fit} here depends not only on parameters $\boldsymbol{p}$, but on the function $\hat{c}(\boldsymbol{p})$.

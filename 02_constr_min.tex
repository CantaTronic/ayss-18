% The problem of minimizing functionals with constraints arises, for example, in the task of kinematic fitting.
% % \begin{block}{Kinematic fitting}
% \begin{itemize}
% \item Tracking detectors provide the coordinates of the triggered sensitive elements along with their errors;
% \item Track-finding involves fitting the particle trajectories to these coordinates;
% \item Sometimes, when the reaction channel is known, the additional information on kinematics could be utilized in terms of
% % \begin{description}
% \item[conservation laws:] $\sum E_\mathrm{initial} = \sum E_\mathrm{final}$, $\displaystyle\sum\vec{P}_\mathrm{initial} = \sum\vec{P}_\mathrm{final}$;
% \item[missing mass:] $\displaystyle\left|\sum\boldsymbol{P}^{(4)}_\mathrm{initial} - \sum\boldsymbol{P}^{(4)}_\mathrm{final}\right|^2 = M_X^2$;
% \end{itemize}
%
% % \end{description}
% This is called \emph{kinematic fitting}.

Поставленная задача нахождения минимума функционала \eqref{track_fit} может быть сформулирована как задача условной минимизации в канонической форме, с учетом известных законов сохранения \eqref{cons_full}--\eqref{cons_miss}, выраженных в виде:
\begin{equation}
\label{eq:constr}
\left\{
\begin{aligned}
\phi_1(P) &= 0,\\
\cdots\\
\phi_n(P) &= 0.
\end{aligned}
\right.
\end{equation}
предмет интереса представляют случаи, когда система \eqref{eq:constr} является нелинейной и её решение ``в лоб'' является невозможным либо нецелесообразным в условиях заданных внешних ораничений.
Возникает необходимость не потерять важную информацию и провести расчёты максимально точно, при этом уложившись в ограничения по времени и трудоемкости работы.

Первые идеи по преодолению подобных сложностей в задачах кинематического фита были предложены в 60-е. Солмицом и Бёрджем \cite{b1} было предложено применение метода множителей Лагранжа (см. далее), который до сих пор находит широкое применение в задачах подобного типа \cite{b4}. %сюда же можно ещё ссылки на эксперименты, там тоже сплошнй Лагранж
Примерно в то же время специалистами ОИЯИ было предложено применение минимизации с тяжёлым членом \cite{b5} (будет описано далее) и разработан пакет для минимизации $\chi^2$-функционалов FUMILI~\cite{fum_1st}.
В данной статье нам интересно сравнить данные методы с 
Использование метода устранения производных для кинфита со связями, было предложено одним из соавторов данной работы.

TODO: подвести к eloimination of differentials.

%%%%%%%%%%%%%%%%%%%%%%%%%% EDP Science %%%%%%%%%%%%%%%%%%%%%%%%%%%%
%
%%%\documentclass[option]{webofc}
%%% "twocolumn" for typesetting an article in two columns format (default one column)
%
\documentclass{webofc}
\usepackage[varg]{txfonts}   % Web of Conferences font
\usepackage{color}
%
\begin{document}
  %
\title{FUMILI-based minimization with constraints using method of elimination of differentials}

\author{
  \firstname{Vladimir} \lastname{Kurbatov},~
  \firstname{Victoria} \lastname{Tokareva}\thanks{
    \email{tokareva@jinr.ru}
   },~
  \firstname{Dmitry} \lastname{Tsirkov}
}

\institute{Laboratory of Nuclear Problems, Joint Institute for Nuclear Research, \\6 Joliot-Curie, Dubna, Moscow region, 141980, Russia}

%TODO: rewrite the abstract a bit different, if possible (the current version has been written for AYSS-18 abstracts and published already)

\abstract{
Some minimization problems, in addition to usual constant limits for single parameters, could imply constraints, i.e. additional relations between parameters $P_1, …, P_n$ in form of equations $\varphi(P_1, \dots, P_n) = 0$.
Often these equations are non-linear and complicated, and thus it is impossible or impractical to eliminate redundant parameters directly. A notable example of such problem is kinematic fitting.
A minimization approach called a method of elimination of differentials is being developed at JINR as an extension to the FUMILI minimizer. It is being used to perform kinematic fitting while analyzing data collected at the ANKE spectrometer.
The talk will cover the mathematical principles of the method, its software implementation and API, and present some examples of its usage.
}
%
\maketitle
   %DONE

% ОБЩИЙ ПЛАН СТАТЬИ
%
% 1 мотивация: кинфит - это очень очень нужно и важно
% 2 кинфит делается посредством минимизации со связями
% 3 история вопроса: два метода
% 4 в ОИЯИ был предложен третий метод - очень кратко суть метода -  как расширение минимизатора ФумилиЁ разработанного так же с ОИЯИ
% 5 минимиатор Фумили хорош тем-то и тем-то (по сравнению с аналогами)
% 6 метод со связями хорош тем-то и тем-то
% 7 программная реализацияЁ как и почему
% 8 пример игрушечный
% 9 пример настоящий
% 10 планы и заключение - "спасти человечество"Ё не меньше))
% 11 библиография

% Фитирование - это настройка весов вычислительной модели таким образом, чтобы она максимально удовлетворяла реальным данным, полученным в эксперименте.
% При этом чем больше мы имеем данных о физике происходящего процесса, тем более адекватные значения параметров мы сможем поулчить в результате подгонки.
% Эти данные, будучи выражены в форме уравнений типа ``кракозябра = 0'', называются связями (constraints).
% Использование такой, более точной, аппроксимации востребовано во многих областях физики. Одной из таких областей является кинематическ


\section{Motivation}
\section{Kinematic fitting}
\section{Cosntraint minimization aproaches}
  \subsection{Method of Lagrange multipliers}
  \subsection{Penalty function method}
  \subsection{Method of ellimination of differentials}
%     \textcolor{red}{TODO: привести к общему знаменателю обозначения в формуле с обозначениями в остальной статье!!!}

% обозначим через $np$ число параметров, от которых зависит функционал \eqref{track_fit}, $nc$ - xbckj
В близи фиксированной точки $P_0$ функционал \eqref{track_fit} может быть представлен степенным рядом
% In the neighborhood of a point $x_0$ the functional $F(x)$ could be expressed as
\begin{equation}\label{func_diff}
\begin{aligned}
F(P_0+\Delta P) &= F(P_0) + \sum_{i=1}^n \frac{\partial F(P_0)}{\partial P_i} \Delta P_i + \frac{1}{2}\sum_{i=1}^n\sum_{j=1}^n \Delta P_i \frac{\partial^2 F(P_0)}{\partial P_i \partial P_j} \Delta P_j\\
&= F(P_0) + G(P_0) \Delta P + \frac{1}{2}\Delta P^T Z(P_0) \Delta P,
\end{aligned}
\end{equation}
Данную формулу можно переписать в матричном виде как:

\begin{equation}
 \label{eq_diff1}
x^2
\end{equation}


and the constraints $\Phi(x)$ as
\begin{equation}\label{constr_diff}
\Phi(x_0+\Delta x) =
\left[
\begin{aligned}
\phi_1(x_0) +& \sum_{i=1}^n \frac{\partial \phi_1(x_0)}{\partial x_i}\Delta x_i\\
&\cdots\\
\phi_m(x_0) +& \sum_{i=1}^n \frac{\partial \phi_m(x_0)}{\partial x_i}\Delta x_i\\
\end{aligned}
\right]
= \Phi(x_0) + D(x_0) \Delta x.
\end{equation}

In the matrix equation $\Phi(x) = \Phi(x_0) + D(x_0) \Delta x$ the rectangular matrix $D$ has $m$ rows and $n$ columns (we have $n$ parameters and $m$ constraints).

The vector $\Delta x$ could be split into $\Delta x_c$ that has $m$ components, and $\Delta x_f$ that has $n-m$ components; the same could be done with the matrix $D$. Then
\begin{equation}\label{phi_split}
\Phi(x) = \Phi(x_0) + D_c(x_0) \Delta x_c + D_f(x_0) \Delta x_f.
\end{equation}
Since \eqref{phi_split} is a linear equation, it could be solved in relation to $\Delta x_c$, resulting in
\begin{equation}\label{X_c}
\Delta x_c = \mathbf{v} + M \Delta x_f.
\end{equation}
Returning to the initial functional \eqref{func_diff}
\[F(x) = F(x_0) + G(x_0) \Delta x + \frac{1}{2}\Delta x^T Z(x_0) \Delta x,\]
we could employ \eqref{X_c} to eliminate the sub-vector $\Delta x_c$, and obtain a similar functional, in contrast depending on only $n-m$ increments $\Delta x_f$:
\begin{equation}\label{func_fin}
F(x) = F'(x_0) + G'(x_0) \Delta x_f + \frac{1}{2}\Delta x_f^T Z'(x_0) \Delta x_f.
\end{equation}

\section{Fumili minimizer and Fumili-based realization of MEDiff}
  %программная реализация будет здесь же
\section{Example: Monte-Carlo simulation}
\section{Kinematic fitting at ANKE}
\section{Conclusion and outlook}

\begin{thebibliography}{00}
% \bibitem{b1} J.P.~Berge, F.T.~Solmitz, H.D.~Taft, Rev.\ Sci.\ Instr.\ \textbf{32} 538 (1961)
% \bibitem{b2} R. Bock, CERN 60-30 (1960)
% \bibitem{b3} G.A. Korn, T.M. Korn, Mathematical handbook for scientists and engineers, McGraw-Hill (1968) 333--335; \foreignlanguage{russian}{Г. Корн, Т. Корн, Справочник по математике для научных работников и инженеров, Наука (1974) 334--336}
% \bibitem{b4} P. Avery, KWFIT package\\ URL: \texttt{http://www.phys.ufl.edu/\textasciitilde{}avery/kwfit/}
% \bibitem{b5} V.I. Moroz, JINR communications R-1958 (1965)\\ URL: \texttt{http://nu73-73.jinr.ru/\textasciitilde{}cyrkov/MorozJINRCommR1958.pdf}
% % \texttt{https://drive.google.com/file/d/1z1hPsccoyXYGA-DT6-7xmeKnYBwPJCS8}
% \bibitem{b6} V.S. Kurbatov, I.N. Silin, Nucl. Instrum. Meth. Phys. Res. A 345 (1994) 346--350
% % \bibitem{b7} S.N. Dymov et al., Nucl. Instrum. Meth. Phys. Res. A 440 (2000) 431--437
% \bibitem{b8} S. Barsov et al., Nucl. Instrum. Meth. Phys. Res. A 462 (2001) 364
% \bibitem{b9} V. Komarov et al., Phys. Rev. C 93 (2016) 065206
% \bibitem{b10} S. Dymov et al., FdModule framework\\ URL: \texttt{http://nu73-73.jinr.ru/\textasciitilde{}dymov/FdModule/index.html}

% % Zainon, R., Butler, A. P. H., Cook, N. J., Butzer, J. S., Schleich, N., de Ruiter, N., Tlustos, L., Clark, M. J., Heinz, R. & Butler, P. H. (2010). Construction and operation of the MARS-CT scanner,
% % Internetworking Indonesia Journal 2(1): 3–10.
% \bibitem{mars}
% R.~Zainon, A.P.H.~Butler \textit{et al.}, Internetworking Indonesia Journal \textbf{2(1)}, 3--10 (2010).
%
% \bibitem{dicom}
% DICOM standard official site, available at http://dicom.nema.org/standard.html.
%
% \bibitem{turbel}
% H.~Turbell, \textit{Cone-beam reconstruction using filtered back-projection}, Link\"oping University Electronic Press, 189 (2001).
% %
% % and use \bibitem to create references.
% \bibitem{trends}
% I.~Scholl, T.~Aach, T.M.~Deserno \textit{et al.}, Comput.\ Sci.\ Res.\ Dev.\ \textbf{26}, 5--13 (2011).
%
% \bibitem{Roobot}
% C.A.~Roobottom, G.~Mitchell, G.~Morgan-Hughes, Clin.\ Radiol.\ \textbf{65 (11)}, 859--67 (2010).
%
% \bibitem{popularCT}
% T.~Hiroyasu, Y.~Minamitani, M.~Yoshimi, M.~Miki, WORLDCOMP'11 \textbf{2011-MRS-84 (5)} (2011).
%
% \bibitem{hierarhCT}
% S.~Xiao, Y.~Bresler, D.C.~Munson, Proceedings 2003 International Conference on Image Processing \textbf{2}, II–819–22 (2003).
%
% \bibitem{77}
% L.~Feldkamp, L.~Davis, J.~Kress, Journal of the Optical Society of America A \textbf{1(6)}, 612--619 (1984).
%
% \bibitem{bigObserve}
% A.~Belle, BioMed Research International \textbf{2015}, 16 (2015).
%
% % \bibitem{Scherl}
% % H.~Scherl, \textit{Evaluation of State-of-the-Art Hardware Architectures for Fast Cone-Beam CT Reconstruction} (Springer Fachmedien Wiesbaden GmbH, Wiesbaden, 2011) XX, 138
% %
% % \bibitem{Yu}
% % L.~Zhou, J.~Chang, Medical Physics \textbf{39} 4000 (2012)
% %
% % \bibitem{Ramani}
% % S.~Ramani, J.A.~Fessler, IEEE Transactions on Medical Imaging, \textbf{31 (3)}, 677 (2012)
% %
% % \bibitem{Yan}
% % G.~Yan, S.~Zhu, Y.~Dai, C.~Qin, Journal of X-Ray Science and Technology \textbf{16 (626)} 225--234 (2008).
%
% \bibitem{76}
% B.~Meng, G.~Pratx, L.~Xing, Med.\ Phys.\ \textbf{38(12)} 6603--6609 (2011).
%
% % \bibitem{77}
% % C.F.~Mackenzie, P.~Hu, A.~Sen et al., AMIA Annual Symposium Proceedings \textbf{2008} 318--322 (2008)
%
% \bibitem{chinese}
% C.-T.~Yang, C.-T.~Kuo, W.-H.~Hsu, W.-C. Shih, Proceedings of the 7th international conference on Advances in Grid and Pervasive Computing \textbf{7296}, 338--349 (2012).
%
% \bibitem{HDFS} HDFS documentation, available at http://hadoop.apache.org/docs/current/hadoop-project-dist/hadoop-hdfs/HdfsDesign.html.


%###################################################################################
% \bibitem{RefJ}
% Format for Journal Reference
% Journal Author, Journal \textbf{Volume}, page numbers (year)
% Format for books
% \bibitem{RefB}
% Book Author, \textit{Book title} (Publisher, place, year) page numbers
% etc
\end{thebibliography}
\end{document}

% % Penalty-function method}
% \begin{itemize}
% \item Proposed in JINR in mid-sixties, see V.I. Moroz, JINR communications R-1958 (1965);
% \item Adds a so-called ``heavy term'' to the minimized functional, designed in a way that values of constraint functions approach zero as this term approaches infinity:
% \[
% \tilde{\Psi}(x) = F(x) + T\sum_{i=1}^{m}\phi_i^2(x),~T \rightarrow \infty.
% \]
% \item The method is very robust and almost always converges, which could be both a benefit (you won't miss a minimum) and a drawback (you should carefully control that your minimum is reasonable).
% \end{itemize}

%
The method of kinematic fitting employing a ``heavy term'', proposed by Moroz~\cite{b5}, is based on the method of penalty functions for solving conditional optimization problems.
% Метод кинематического фита с тяжелым членом, предложенный В.И.Морозом, основан на методе штрафных функций решения задач условной оптимизации.

The essence of the method is to reduce the task of minimizing the function \eqref{track_fit} with the conditions \eqref{eq:constr} to the problem of finding the unconditional minimum of the function
% Суть метода состоит в сведении задачи минимизации функции \eqref{track_fit} при выполнении условий \eqref{eq:constr} к задаче поиска безусловного минимума функции
\begin{equation}
\hat{F}(\boldsymbol{p}, \boldsymbol{c}) = F(\boldsymbol{p}, \boldsymbol{c}) + T\sum_{k=1}^{m}\phi_k^2(\boldsymbol{p}),~T \rightarrow \infty.
\end{equation}

% При этом функцию $\tilde{\Psi}(P, c)$ называют штрафной или функцией с тяжёлым членом $T$.
In this case, the function $\hat{F}(\boldsymbol{p}, \boldsymbol{c})$ is called a penalty function or a function with a ``heavy term'' $\displaystyle T\sum_{k=1}^{m}\phi_k^2(\boldsymbol{p})$.
% При такой постановке при выходе за пределы, задаваемые ограничениями \eqref{eq:constr}, слагаемое ``с тяжёлыми членом'' будет увеличивать значение минимизируемой функции, таким образом гарантируя, что оптимальное значение может быть найдено только в пределах области ограничений.
With this formulation of the problem, if we exceed the limits specified by the constraints \eqref{eq:constr}, the ``heavy term'' will increase the value of the minimized functional, thus ensuring that the optimal value can be found only within the constraint area.

% Минимизация штрафной функции может выполняться с использованием любого метода оптимизации, например, градиентного спуска.
The minimization of the penalty function can be performed using any optimization method, for example, gradient descent.

% % Penalty-function method}
% \begin{itemize}
% \item Proposed in JINR in mid-sixties, see V.I. Moroz, JINR communications R-1958 (1965);
% \item Adds a so-called ``heavy term'' to the minimized functional, designed in a way that values of constraint functions approach zero as this term approaches infinity:
% \[
% \tilde{\Psi}(x) = F(x) + T\sum_{i=1}^{m}\phi_i^2(x),~T \rightarrow \infty.
% \]
% \item The method is very robust and almost always converges, which could be both a benefit (you won't miss a minimum) and a drawback (you should carefully control that your minimum is reasonable).
% \end{itemize}

%

Метод кинематического фита с тяжелым членом, предложенный В.И.Морозом, основан на методе штрафных функций решения задач условной оптимизации.

Суть метода состоит в сведении задачи минимизации функции \label{track_fit} при выполнении условий \eqref{eq:constr} к задаче поиска безусловного минимума функции
\begin{equation}
 \tilde{\Psi}(P, c) = F(P, c) + T\sum_{k=1}^{K}\phi_k^2(P),~T \rightarrow \infty.
\end{equation}

При этом функцию $ \tilde{\Psi}(P, c)$ называют штрафной или функцией с тяжёлым членом $T$. При такой постановке при выходе за пределы, задаваемые ограничениями \eqref{eq:constr}, слагаемое ``с тяжёлыми членом'' будет увеличивать значение минимизируемой функции, таким образом гарантируя, что оптимальное значение может быть найдено только в пределах области ограничений.

Минимизация штрафной функции может выполняться с использованием любого метода оптимизации, например, градиентного спуска.
